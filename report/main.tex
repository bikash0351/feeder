% IOE MSC Latex Template by
% Santosh Giri, Assistant Professor, DOECE, Pulchowk Campus, IOE, TU
\documentclass[12pt,oneside]{report}
\usepackage{ragged2e}
\usepackage{xcolor}
\usepackage[utf8]{inputenc}
\usepackage{natbib}%referencing
\usepackage[left=1.0in,right=1.0in,top=.8in,bottom=.8in]{geometry}
\usepackage{float}
\linespread{1.3}
\usepackage{graphicx}%figures
\usepackage{rotating}%landscape
\usepackage{amsmath}%math
\usepackage{titlesec} %formatting chapters
%\titlespacing*{<command>}{<left>}{<before-sep>}{<after-sep>}
\titlespacing*{\chapter}{-15pt}{10pt}{15pt}
\titlespacing*{\section}{0pt}{0pt}{5pt}
\titlespacing*{\subsection}{0pt}{5pt}{5pt}
\titleformat{\chapter}[hang]
{\normalfont\huge\bfseries}{\chaptertitlename\ \thechapter.}{1em}{}
\renewcommand{\chaptername}{}
\graphicspath{{Images/}}%image folder name
%Cover page contents
\title{
{\includegraphics[scale=.3]{logotu.jpg}}\\
{\large\uppercase{
    Tribhuvan University\\
    Institute of Engineering\\
    Pulchowk Campus\\
    \vspace{.5cm}
    A Thesis Report On\\Short-Term Electrical Load Forecasting for\\Baneshwor Feeder Using Machine and Deep Learning Models\\
    \vspace{.5cm}
    \textbf{Submitted By:}\\Sujit Koirala\\(PUL075MSPSE016)\\
    \vspace{.5cm}
    \textbf{Supervised By:}\\ Prof. Dr. Rogers s Pressman\\
    \vspace{.5cm}
    \textbf{Submitted To:}\\ Department of Electrical Engineering
}}}
\date{December, 2025} %{Month Year}
\begin{document}
\maketitle

\pagenumbering{roman}
\setcounter{page}{2}
\chapter*{Declaration}
\addcontentsline{toc}{chapter}{\numberline{}Declaration}
I hereby declare that this study/research entitled \textbf{[Put the title of the project/thesis here....]} is based on our original research work. Related works on the topic by other researchers have been duly acknowledged. I owe all the liabilities relating to the accuracy and authenticity of the data and any other information included hereunder.
\vspace{2cm}\\
\textbf{Name of the Student ( Roll no)}\\
\newline
Date:
\chapter*{Recommendation}
\addcontentsline{toc}{chapter}{\numberline{}Recommendation}
This is to certify that this project report entitled \textbf{[Put the title of the project/thesis here]} prepared and submitted by \textbf{[Put the name \& roll no of the student here]}, in partial fulfillment of the requirements of the Master degree of Engineering in......, awarded by Tribhuvan University, has been completed under my/our supervision. I/we recommend the same for acceptance by Tribhuvan University.
\vspace{75pt}\\
-----------------------------------\\
Name of the Supervisor: ABC\\
Designation: ABC\\
Organization: ABC\\
Date: ABC\\
\vspace{20pt}\\
-----------------------------------\\
Name of the Co-supervisor: ABC\\
Designation: ABC\\
Organization: ABC\\
Date: ABC\\
\chapter*{Page of Approval}
\addcontentsline{toc}{chapter}{\numberline{}Page of Approval}
\vspace{1.5cm}
\begin{center}
\uppercase{
Tribhuvan Universiy\\
Institute of Engineering\\
Pulchowk Campus\\
Department of Electronics and Computer Engineering
    }
    \end{center}
\vspace{.5cm}
This project/thesis entitled \textbf{[Put the title of the project/thesis here]}
prepared and submitted by \textbf{[Put the Name and Roll no of the student here]} has been examined
by us and is accepted for the award of the Master's degree in [Put name of the program]
by Tribhuvan University.

\vspace{2cm}
\begin{minipage}{.5\textwidth}
        \raggedright
    .............................\\
    Supervisor\\
    \textbf{Name}\\
    Designation\\
    Organization Name and Address.\\
    \end{minipage}%
    \begin{minipage}{0.5\textwidth}
    \raggedleft
    .............................\\
    External examiner\\
    \textbf{Name}\\
    Designation\\
    Organization Name and Address.\\
    \end{minipage}\\
    \vspace{.8cm}
    \begin{center}
    .............................\\
    Co-Supervisor (if Any)\\
    \textbf{Name}\\
    Designation\\
    Organization Name and Address.\\
    \vspace{.6cm}
    \end{center}
\centerline{Date of approval:}
%new chapter
\chapter*{Copyright}
\addcontentsline{toc}{chapter}{\numberline{}Copyright}
The author has agreed that the Library, Department of Electronics and Computer Engineering, Pulchowk Campus, and Institute of Engineering may make this report available for inspection. Moreover, the author has agreed that permission for extensive copying of this project report for scholarly purposes may be granted by the supervisors who supervised the work recorded herein or, in their absence, by the Head of the Department wherein the project report was done. It is understood that recognition will be given to the author of this report and the Department of Electronics and Computer Engineering, Pulchowk Campus, Institute of Engineering for any use of the material of this project report. Copying publication or the other use of this report for financial gain without the approval of the Department of Electronics and Computer Engineering, Pulchowk Campus, Institute of Engineering, and the author's written permission is prohibited.\\
Request for permission to copy or to make any other use of the material in this report in whole or in part should be addressed to:\\
\newline
\newline
\newline
Head\\
Department of Electronics and Computer Engineering\\
Pulchowk Campus, Institute of Engineering, TU\\
Lalitpur, Nepal.
%new chapter
\chapter*{Acknowledgments}
\addcontentsline{toc}{chapter}{\numberline{}Acknowledgements}
I would like to express my sincere gratitude to my supervisor and faculty members of the
Department of Electrical Engineering for their valuable guidance, continuous support, and
encouragement throughout the course of this project. Their technical insights and
constructive feedback were instrumental in shaping this work.
I am also thankful to the Nepal Electricity Authority and relevant data-providing institutions
for making the load and meteorological data available for this study. Their cooperation
greatly contributed to the successful completion of the analysis.
Special thanks go to my friends and colleagues for their support, discussions, and motivation
during the project period. Finally, I would like to express my heartfelt appreciation to my
family for their constant encouragement and support throughout my academic journey.
\vspace{50pt}\\
\textbf{Sujit Koirala (PUL075MSPSE016)}\\
%new chapter
\chapter*{Abstract}
\noindent Accurate short-term electrical load forecasting plays a crucial role in the efficient planning
and operation of modern power systems. With increasing load variability influenced by
weather conditions, temporal patterns, and socio-economic activities, traditional statistical
methods often struggle to capture complex and nonlinear demand behavior. This project
focuses on short-term electrical load forecasting for the Lekhnath Feeder using machine
learning--based approaches.

\noindent Historical hourly load data, along with meteorological variables such as air temperature,
global solar radiation, and relative humidity, were used to develop predictive models.
Comprehensive data preprocessing was performed, including missing value imputation,
outlier treatment, temporal feature extraction, and cyclical encoding of time-based
variables. Several machine learning models were implemented and evaluated, including
Linear Regression, Ridge Regression, Support Vector Regression, Random Forest, Gradient
Boosting, and XGBoost. Hyperparameter tuning was applied to improve model
performance.

\noindent The models were assessed using standard evaluation metrics such as Mean Absolute Error
(MAE), Root Mean Square Error (RMSE), Mean Absolute Percentage Error (MAPE), and
R-squared ($R^2$). The results show that ensemble-based models, particularly tuned
XGBoost and Random Forest models, significantly outperform linear and baseline
methods. The findings highlight the effectiveness of machine learning techniques for
feeder-level short-term load forecasting and provide valuable insights for operational
planning and decision-making in power distribution systems.\\
\newline
Keywords: \textit{short-term load forecasting, machine learning, XGBoost, Random Forest, power distribution systems}
\addcontentsline{toc}{chapter}{\numberline{}Abstract}
\tableofcontents
\addcontentsline{toc}{chapter}{\numberline{}Contents}
\listoffigures
\addcontentsline{toc}{chapter}{\numberline{}List of Figures}
\listoftables
\addcontentsline{toc}{chapter}{\numberline{}List of Tables}
\chapter*{List of Abbreviations}
\addcontentsline{toc}{chapter}{\numberline{}List of Abbreviations}
\begin{tabular}{c l}
\textbf{NEA}     &  Nepal Electricity Authority\\
\textbf{STLF}    &  Short-Term Load Forecasting\\
\textbf{ML}      &  Machine Learning\\
\textbf{DL}      &  Deep Learning\\
\textbf{RNN}     &  Recurrent Neural Network\\
\textbf{LSTM}    &  Long Short-Term Memory\\
\textbf{GRU}     &  Gated Recurrent Unit\\
\textbf{MLP}     &  Multi-Layer Perceptron\\
\textbf{SVR}     &  Support Vector Regression\\
\textbf{RF}      &  Random Forest\\
\textbf{GBR}     &  Gradient Boosting Regressor\\
\textbf{XGBoost} &  Extreme Gradient Boosting\\
\textbf{MAE}     &  Mean Absolute Error\\
\textbf{MSE}     &  Mean Squared Error\\
\textbf{RMSE}    &  Root Mean Squared Error\\
\textbf{MAPE}    &  Mean Absolute Percentage Error\\
\textbf{SMAPE}   &  Symmetric Mean Absolute Percentage Error\\
\textbf{R\textsuperscript{2}}      &  Coefficient of Determination\\
\textbf{MW}      &  Megawatt\\
\textbf{BS}      &  Bikram Sambat (Nepali Calendar)\\
\textbf{AD}      &  Anno Domini (Gregorian Calendar)\\
\textbf{EDA}     &  Exploratory Data Analysis\\
\textbf{IQR}     &  Interquartile Range\\
\textbf{TFT}     &  Temporal Fusion Transformer\\
\textbf{API}     &  Application Programming Interface
\end{tabular}
\chapter*{List of units and conversions}
\addcontentsline{toc}{chapter}{\numberline{}List of Units and conversions}
\begin{tabular}{l l}
$m^3$   &   Meter cube (Cubic meter)\\
Sq.ft   &   Square feet\\
add     &   more
\end{tabular}
\chapter{Introduction}
\pagenumbering{arabic}%start arabic numbering(1,2,3..) from here
This project focuses on short-term electrical load forecasting at the feeder level using data-driven machine learning and deep learning techniques. Historical load data combined with weather and temporal features were used to model and predict hourly power demand. Multiple forecasting models were developed and evaluated to identify the most effective approach for accurate and reliable load prediction.
\section{Background}
Electricity demand is never constant. It rises and falls with daily routines, temperature changes, business hours, and countless other factors. For a power system operator, being able to predict this demand even just a few hours ahead can make a huge difference. Accurate short-term forecasting helps optimize generation schedules, reduce operational costs, manage peak hours more confidently, and maintain a reliable supply.

Short-Term Load Forecasting (STLF) typically focuses on horizons ranging from one hour to a day ahead. These forecasts are critical for economic dispatch, unit commitment, load flow analysis, and real-time operation. Traditionally, utilities relied on statistical approaches such as linear regression, ARIMA, exponential smoothing, and Holt-Winters. These techniques can work well when patterns are simple, but they struggle with real-world load curves that are nonlinear, noisy, and influenced by many interacting variables.

Machine Learning models like Random Forest, Support Vector Regression, and XGBoost have shown strong results in several energy-related forecasting tasks. Their ability to capture nonlinear relationships makes them a natural fit for electricity load prediction. Likewise, Deep Learning approaches, especially recurrent neural networks such as LSTM and GRU, can learn temporal dependencies more effectively than traditional models.

The Baneshwor Feeder of the Nepal Electricity Authority serves a mixed group of consumers in the Baneshwor region. Its load pattern reflects residential lifestyles, commercial activity, seasonal tourism impacts, and local weather changes. Daily and weekly cycles are clearly visible, but there are also irregularities that simple models fail to capture. As power consumption continues to grow and diversify, the ability to forecast the feeder's short-term load accurately has become even more important. This creates a strong motivation to investigate how modern ML and DL models can improve forecasting performance for this specific feeder.
\section{Problem Statement}
The current forecasting practices for the Baneshwor Feeder rely heavily on manual estimation or basic statistical techniques. These methods do not fully capture the nonlinear and dynamic nature of the load profile, especially when multiple influencing factors, like temperature, humidity, rainfall, weekends, and special events come into play. As a result, prediction errors tend to increase during peak hours, sudden weather changes, and seasonal transitions.

Inaccurate short-term forecasts have several consequences. They can affect how generation is scheduled, leading to either unnecessary reserve margins or inadequate supply. They may increase operational costs and technical losses at the distribution level. In the worst cases, poor foresight during high-demand periods can create voltage drops, reliability concerns, or inefficient load-shedding decisions.

Despite the availability of historical load and weather data, there has not been a systematic study applying and comparing advanced machine learning and deep learning approaches specifically for the Baneshwor Feeder. The lack of a data-driven forecasting system means operators do not yet benefit from models that are capable of learning complex relationships within the data.

This thesis aims to address these gaps by building a complete forecasting framework using multiple ML and DL models, evaluating their performance, and identifying the most suitable approach for accurate short-term load prediction of the Baneshwor Feeder.
\section{Objectives}
To develop and evaluate machine learning and deep learning models for short-term electrical load forecasting of the Baneshwor Feeder to improve prediction accuracy and operational efficiency.

\begin{enumerate}
    \item To collect and preprocess historical load data and relevant influencing factors such as weather variables and calendar effects for the Baneshwor Feeder.
    \item To analyze consumption patterns and influencing factors (e.g., time, weather, festivals).
    \item To implement various machine learning models, including Support Vector Regression (SVR), Random Forest (RF), and XGBoost, for load forecasting.
    \item To design and train deep learning models such as Long Short-Term Memory (LSTM) and Gated Recurrent Unit (GRU) networks to capture temporal dependencies in load data.
    \item To evaluate and compare the performance of ML and DL models using standard error metrics (e.g., RMSE, MAPE, MAE, MSE, R-squared).
    \item To compare and select the most effective forecasting model.
    \item To recommend the most suitable forecasting model for operational use in the Baneshwor Feeder.
\end{enumerate}
\section{Scope}
This study is geographically limited to the Baneshwor Feeder under the Nepal Electricity Authority. The temporal scope focuses on short-term load forecasting with a prediction horizon of up to 24 hours ahead, utilizing historical hourly load data as the primary foundation for model development. The dataset encompasses historical load data from the Baneshwor Feeder, complemented by weather data including temperature, humidity, and rainfall, along with calendar data distinguishing weekdays, weekends, and holidays.

From a technical perspective, the research implements several machine learning models including Support Vector Regression (SVR), Random Forest, and XGBoost, alongside deep learning architectures such as Long Short-Term Memory (LSTM) and Gated Recurrent Unit (GRU) networks. The performance of these models is rigorously evaluated using standard metrics including Root Mean Squared Error (RMSE), Mean Absolute Percentage Error (MAPE), Mean Absolute Error (MAE), Mean Squared Error (MSE), and the coefficient of determination (R-squared). It should be noted that the accuracy of forecasts is inherently dependent on the quality and completeness of the historical data available. Additionally, this study does not extend to medium-term or long-term forecasting horizons, and the scope explicitly excludes renewable generation forecasting from its analysis.
\section{Limitation}
Despite the promising results obtained in this study, certain limitations were encountered, primarily related to data availability, model assumptions, and scope of analysis.
\begin{enumerate}
    \item \textbf{Dependence on data quality:} Forecast accuracy is limited by the completeness and reliability of the historical load and weather data. Missing values, sensor errors, or inconsistent reporting can influence model performance.
    \item \textbf{Model sensitivity to sudden changes:} Unexpected events such as outages, festivals, abrupt weather shifts, or abnormal consumption patterns are difficult for data-driven models to predict accurately.
    \item \textbf{Deep learning computation constraints:} Training LSTM and GRU models requires more computational resources and time compared to ML models. Their performance may vary depending on the hardware used.
    \item \textbf{Limited feature diversity:} Although weather and calendar data are included, other influential factors like economic activities, special events, or industrial load profiles are not part of the dataset.
    \item \textbf{Generalization across feeders:} The models developed in this study are tailored specifically to the Baneshwor Feeder and may not generalize directly to other feeders without retraining or adaptation.
\end{enumerate}
\chapter{Literature Review}
The literature review presents a comprehensive overview of existing research related to short-term electrical load forecasting using machine learning and deep learning techniques. It examines previously published studies to understand commonly used methodologies, datasets, and performance evaluation approaches in the domain of power system load forecasting. Reviewing prior work helps to identify current research trends, strengths, and limitations of existing models, while also highlighting gaps that motivate the need for this study. By situating the present research within the context of established knowledge, this chapter provides a foundation for model selection and methodological design adopted in this work.
\section{Related work}
There are many previous works done for electrical load forecasting from short-term electrical load forecasting, to medium-term and long-term. Most of the studies have done short-term load forecasting.

\subsection{Machine Learning Approaches}
Singla et al. (2019) employed Artificial Neural Networks for 24-hour short-term load forecasting, utilizing dew point temperature, dry bulb temperature, and humidity as input features. Their work demonstrated the effectiveness of ANN in capturing the relationship between weather variables and electrical load demand. Similarly, Desai et al. (2021) utilized the Prophet model from Meta to perform short-term load forecasting, incorporating time, temperature, humidity, and weather forecast data as features. The Prophet model's ability to handle seasonal patterns and missing data made it suitable for load forecasting applications.

Matrenin et al. (2022) conducted a study on medium-term load forecasting using ensemble machine learning models. They compared XGBoost and AdaBoost against traditional methods including SVR, decision trees, and Random Forest. Their results highlighted the superior performance of gradient boosting techniques for capturing complex load patterns. Aguilar Madrid \& Antonio (2021) tested five machine learning models and found XGBoost to be the most accurate for predictions, using historical load data, weather information, and holiday indicators as input features. Their comprehensive evaluation demonstrated XGBoost's ability to handle diverse feature sets effectively.

Guo et al. (2021) analyzed three popular ML methods for load forecasting: Support Vector Machine, Random Forest, and LSTM. They proposed a fusion forecasting approach that combined outputs from all three models, demonstrating that ensemble methods could improve prediction accuracy beyond individual model performance. Saglam et al. (2024) performed a comparison between optimization methods (Particle Swarm Optimization, Dandelion Optimizer, Growth Optimizer) and machine learning models (SVR, ANN) for instantaneous peak electrical load forecasting. They found that ANN combined with Growth Optimizer outperformed other models and identified a strong positive correlation between GDP and peak load demand.

Jain \& Gupta (2024) conducted a comprehensive evaluation of various machine learning algorithms for power load prediction, including Support Vector Machines, LSTM, ensemble classifiers, and Recurrent Neural Networks. Their study emphasized the importance of data preprocessing methods, feature selection strategies, and performance assessment metrics in achieving accurate forecasts. The research demonstrated that ensemble methods and deep learning approaches consistently outperformed traditional statistical models.

\subsection{Deep Learning Architectures}
Chapagain et al. (2021) explored time series regression along with machine learning and deep learning models for electricity demand forecasting in Kathmandu Valley. They found LSTM demonstrating outstanding performance in terms of MAPE and RMSE, using deterministic variables such as day type and temperature. Their work validated the effectiveness of recurrent architectures for capturing temporal dependencies in load data.

Acharya et al. (n.d.) performed short-term electrical load forecasting for the Gothatar feeder using six input features. They found that Recurrent Neural Networks outperformed baseline methods including Single Exponential Smoothing, Double Exponential Smoothing, and Holt-Winter's method. This study confirmed that RNNs could better model the nonlinear and time-dependent characteristics of feeder-level load patterns.

Cordeiro-Costas et al. (2023) conducted a comprehensive comparison of load forecasting methods, including Random Forest, SVR, XGBoost, Multi-Layer Perceptron, and LSTM. They also explored Conv-1D models and found that LSTM achieved the lowest error rates across multiple evaluation metrics. Their research highlighted the trade-off between model complexity and forecasting accuracy in practical applications.

Dong et al. (2024) provided a comprehensive survey on deep learning-based short-term electricity load forecasting covering the past decade. They examined the entire forecasting process, including data preprocessing, feature extraction, deep learning modeling and optimization, and results evaluation. The survey identified CNN-LSTM hybrid architectures as widely adopted solutions due to exceptional performance in capturing both spatial and temporal features. Their analysis revealed that most recent studies focused on short-term horizons ranging from one hour to several days ahead.

\subsection{Hybrid and Advanced Architectures}
Wen et al. (2024) proposed a hybrid deep learning model combining Gated Recurrent Units and Temporal Convolutional Networks with an attention mechanism for short-term load forecasting. The GRU captured long-term dependencies in time series data, while TCN efficiently learned patterns and features. The attention mechanism automatically focused on input components most relevant to the prediction task, significantly enhancing model performance. Their approach demonstrated superior accuracy compared to standalone architectures.

Alhussein et al. (2020) developed a hybrid CNN-LSTM framework for short-term individual household load forecasting. The model used CNN layers for feature extraction from input data and LSTM layers for sequence learning. Evaluated on the Smart Grid Smart City dataset, the hybrid model achieved an average MAPE of 40.38\%, outperforming standalone LSTM models that obtained 44.06\% MAPE. This work demonstrated the effectiveness of combining convolutional and recurrent architectures for handling high volatility in household-level load data.

Hasanat et al. (2024) proposed a parallel multichannel network approach using 1D CNN and Bidirectional LSTM for load forecasting in smart grids. Unlike traditional stacked CNN-LSTM architectures that use convolutions as preprocessing steps, their model independently processed spatial and temporal characteristics through parallel channels. The research addressed the issue of temporal feature neglect in existing models and incorporated cyclic features through trigonometric transformations, achieving superior accuracy on diverse building types.

\subsection{Transformer-Based Models}
Chan \& Yeo (2024) proposed a sparse transformer-based approach for electricity load forecasting that addressed the computational complexity limitations of standard transformer architectures. Their model applied sparse attention mechanisms to capture temporal dependencies more efficiently, achieving comparable accuracy to RNN-based state-of-the-art methods while being up to 5 times faster during inference. The model was enhanced to support multivariate inputs including weather data, demonstrating versatility in forecasting loads from individual households to city levels.

Zhang et al. (2022) developed a Time Augmented Transformer model for short-term electrical load forecasting, incorporating temporal features and self-attention mechanisms to capture complex dynamic nonlinear sequence dependencies. Their experimental results showed that multivariate inputs including weather and calendar features produced significantly better predictions than univariate approaches. The attention mechanism's capacity to capture complex dynamical patterns in multivariate data contributed to improved forecasting accuracy.

Lu \& Chen (2024) proposed a multivariate data slicing transformer neural network for load forecasting in power systems with high-penetration renewables. The transformer model excelled in capturing spatiotemporal relationships by modeling global correlations through self-attention mechanisms. Their approach demonstrated superior performance in handling the intermittency and volatility characteristics brought by renewable energy integration, outperforming traditional statistical models and conventional machine learning methods.

\subsection{Comparative Studies and Ensemble Methods}
Banik \& Biswas (2024) developed an enhanced stacked ensemble model combining Random Forest and XGBoost for renewable power and load forecasting. The Random Forest model first forecasted the target variable, followed by XGBoost improving predictions through combination of RF outputs. A meta-model using logistic regression then learned the optimal combination, achieving 99\% accuracy on R² evaluation metrics for both short-term and long-term predictions in Agartala City dataset.

Kwon et al. (2020) conducted extensive research on learning models combined with data clustering and dimensionality reduction for short-term electricity load forecasting. They adapted k-means clustering for data grouping and utilized kernel PCA, UMAP, and t-SNE for dimensionality reduction. Applied to neural network-based models on large-scale electricity usage data from 4,710 households, their approach demonstrated improved forecasting performance through effective data preprocessing and feature engineering.

Nabavi et al. (2024) combined Discrete Wavelet Transform with LSTM to improve electricity load forecasting accuracy. The DWT decomposed load series into multiple frequency components, allowing LSTM to learn from denoised and structured representations. Their research demonstrated that preprocessing techniques significantly enhanced deep learning model performance, particularly for datasets with high noise levels and irregular patterns.
\chapter{Methodology}
This chapter includes a discussion about the way you conducted your thesis research to meet the objectives. The methodology should be summarized in the form of a block diagram /flowchart. The selected method should be discussed in detail along with the justification for selecting the methodology. Each method used in the thesis should be directly and specifically linked with the research objectives. It should include research design (historical research, experimental research, field research, and survey research), research approaches (qualitative, quantitative), study area, study population, sample selection (sample selection methods), sample size, methods of data collection (Key Informant Interview, Focused Group Discussion, questionnaire survey, modeling, observation, measurement) and data analysis approach \& tools.
\chapter{Experimental Setup (if any)}
In this section, you describe how the experiment was done and summarize how the data was taken. One typically describes the instruments and detectors that were used. Describe the procedure that was followed to collect the data etc.
\chapter{Results \& Discussion}
This section should present the findings of the study in logical sequences in line with the specific objectives. The presentation of data and facts should be explained regarding plausibility and compared with data from similar studies. The causal factors behind the findings should be discussed about other variables under consideration in the study based on Focused Group Discussion (FGD), Key Informant Interview (KII), questionnaire survey, modeling, observation, measurement, or literature reviews.
\chapter{Conclusions \& Recommendations}
The conclusion is an integration of various issues covered in the body of the thesis. The conclusion includes noting any implications resulting from the discussion and making policy recommendations and the need for further research. Hence, the conclusion should be a logical ending to what has been previously discussed. It must pull together all parts of the argument and refer the reader back to the focus you have outlined in your introduction and to the central topic. Never present any new information in this section. Thus, the conclusion and recommendation of the study must be limited within the scope of the research.
\chapter{Limitations and Future enhancement}
This chapter should contain the major limitations of the project and the further enhancement of the project/research shortly with a different but related approach. \textbf{Referencing checking here}\cite{giri2019transfer}
\addcontentsline{toc}{section}{References}
\renewcommand{\bibname}{References}
\bibliographystyle{unsrt}
\bibliography{ref}
\chapter*{Appendices}
\addcontentsline{toc}{section}{APPENDICES}
This page contains a data sheet, coding, procedure, photograph, questionnaire, and other essential documents. This page should be started from an odd page and APPENDIX numbering should be A, B, C, etc.
\end{document}
